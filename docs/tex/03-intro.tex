\section*{ВВЕДЕНИЕ}
\addcontentsline{toc}{section}{ВВЕДЕНИЕ}

Автоматизация --- процесс, освобождающий человека от множества механических задач и помогающий ему повысить производительность труда. Автоматизация затронула большое количество процессов в современном мире: от работы кассира в магазине и подсчета стоимости покупок до вычисления значений физических величин в труде ученых. Также автоматически производится подсчет очков на соревнованиях различных дисциплин производится автоматически. Однако существуют организации, судейство созтязаний которых происходит вручную: к примеру, Российская Федерация Подводного Рыболовства (РФПР), которая занимается проведением соревнования по подводной охоте \cite{rfpr}. Деятельность федерации, связанная с регистрацией участников и подсчетом очков, не автоматизирована.

Целью курсовой работы является проектирование и разработка базы данных для проведения соревнований Российской Федерации Подводного Рыболовства.

Для достижения поставленной цели, необходимо решить следующие задачи:
\begin{itemize}[label=---]
	\item определить необходимый функционал приложения, предоставляющего доступ к базе данных;
	\item выделить роли пользователей, а также формализовать данные;
	\item проанализировать системы управления базами данных и выбрать подходящую систему для хранения данных;
	\item спроектировать базу данных, описать ее сущности и связи, спроектировать триггер;
	\item реализовать интерфейс для доступа к базе данных.
\end{itemize}

Итогом работы станет приложение, предоставляющее доступ к базе данных.
\pagebreak