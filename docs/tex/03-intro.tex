\section*{ВВЕДЕНИЕ}
\addcontentsline{toc}{section}{ВВЕДЕНИЕ}

Автоматизация --- процесс, освобождающий человека от множества механических задач и помогающий ему повысить производительность труда. Автоматизация затронула большое количество процессов в современном мире: от работы кассира в магазине и подсчета стоимости покупок до вычисления значений физических величин в работе ученых. Также подсчет очков на соревнованиях различных дисциплин производится автоматически. Однако существуют организации, судейство созтязаний которых происходит вручную: Российская Федерация Подводного Рыболовства (далее: РФПР) занимается проведением соревнования по подводной охоте \cite{rfpr}. Деятельность федерации, связанная с подсчетом очков и ведением архива участников соревнований, не автоматизирована.

Целью курсовой работы является проектирование и разработка базы данных для проведения соревнований Российской Федерации Подводного Рыболовства.

Для достижения поставленной цели, необходимо решить следующие задачи:
\begin{itemize}[label=---]
	\item определить необходимый функционал приложения, предоставляющего доступ к базе данных;
	\item выделить  роли пользователей приложения, а также формализовать данные;
	\item проанализировать системы управления базами данных и выбрать подходящую систему для хранения данных;
	\item спроектировать базу данных, описать ее сущности и связи;
	\item реализовать интерфейс для доступа к базе данных.
\end{itemize}

Итогом работы станет приложение, предоставляющее доступ к базе данных.
\pagebreak