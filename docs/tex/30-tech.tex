\section{Технологический раздел} \label{tech}

В рамках конструкторского раздела курсовой работы будут выбраны средства реализации, будет представлено создание базы данных, ролей и выделение им прав, создание триггера, а также разработан пользовательский интерфейс. Важным этапом является тестирование функционала базы данных и программы, которое позволит оценить эффективность и соответствие требованиям. В данном разделе будет также описан метод тестирования функционала базы данных. 

\subsection{Средства реализации}

В качестве языка программирования выбран Swift \cite{swift} --- мультипарадигмальный язык программирования, разработанный компанией Apple \cite{apple} для создания приложений под iOS, macOS, watchOS и tvOS. 

В качестве среды разработки используется XCode \cite{xcode} --- интегрированная среда разработки для создания приложений для операционных систем iOS, macOS, watchOS и tvOS. Xcode предоставляет инструменты для написания кода на языках Swift и Objective-C, отладки приложений, создания пользовательских интерфейсов, тестирования и сборки приложений. Xcode также содержит симуляторы устройств для тестирования приложений без фактического устройства.

\subsection{Создание базы данных}

В процессе проектирования базы данных были учтены все требования к хранению и управлению информацией в приложении. В конструкторском разделе были выбраны подходящие средства реализации для создания базы данных, а затем была спроектирована сама база данных. В соответствии с выбранной СУБД и спроектированной базой данных было осуществлено создание ее сущностей. Программный код представлен в приложении А, листинг \ref{lst:code1}. 
 
\subsection{Создание ролей и выделение им прав}

Создание ролевой модели позволило определить права доступа к данным для различных пользователей системы. В конструкторском разделе была выделена ролевая модель, которая определяет права доступа к данным для различных пользователей системы.

В модели были выделены следующие роли:
\begin{itemize}[label=---]
	\item участник;
	\item судья;
	\item администратор.
\end{itemize}

Соответствующий этой ролевой модели код выделения прав представлен в листингах \ref{lst:role1} --- \ref{lst:role7}.

\subsection{Создание триггера}

Для обеспечения более гибкой работы с базой данных был создан триггер, который автоматически выполняет определенные действия при изменении определенных данных в базе. Это позволяет ускорить обработку данных и снизить нагрузку на систему. Для создания триггера, обновляющего значения поля score в связанных с уловом сущностях, была написана процедура  scoreTrigger, а также вспомогательные функции updateStepScore, updateParticipantScore, updateTeamScore. В листингах \ref{lst:code3} ---  \ref{lst:code6} приложения А представлен соответствующий программный код.

\subsection{Методы тестирования}

Для тестирования модулей приложения, в том числе уровня доступа к базе данных, использовался XCTest \cite{xctest} --- фреймворк для модульного тестирования приложений, написанных на языках Swift и Objective-C, в Xcode. Он предоставляет разработчикам возможность создавать и запускать автоматические тесты, чтобы убедиться в корректности работы функций отдельных частей приложения. XCTest включает в себя множество методов для создания и выполнения проверок, а также для анализа результатов испытаний. 

Так, например, в листинге \ref{lst:cod} представлен код unit--теста для проверки корректности работы функции добавления сущности в базу данных.

\begin{lstlisting}[label=lst:cod, caption=Код unit--теста]
func testCreateLoot() throws {
	let loot = Loot(id: "6442852b2b74d595cb4f4764", fish: "Щука", weight: 500, score: 1000)
	var createdLoot: Loot?
        
	XCTAssertNoThrow(createdLoot = try lootRepository.createLoot(loot: loot))
	XCTAssertEqual(createdLoot, loot)
}
\end{lstlisting}

\subsection{Пользовательский интерфейс}

Важным элементом разработки приложения является пользовательский интерфейс. Был разработан удобный и интуитивно понятный интерфейс, который обеспечивает удобный доступ к данным и удовлетворяет потребности пользователей. Пользователи могут легко находить необходимую информацию и выполнять нужные действия. Для взаимодействия с базой данных был разработан интерфейс в виде мобильного приложения для iOS, представленный на рисунках \ref{fig:gui1} ---  \ref{fig:gui7} приложения Б.


\subsection{Вывод из раздела}

Одним из ключевых элементов данного раздела является создание базы данных. Для этого ранее были выбраны подходящие средства реализации, в рассмотренном же разделе была создана сама базы данных, которая учитывает все требования к хранению и управлению информацией в приложении. Кроме того, была выделена ролевая модель, которая определяет права доступа к данным для различных пользователей системы. Для обеспечения более гибкой работы с базой данных был создан триггер, который автоматически выполняет определенные действия при изменении определенных данных в базе. Это позволяет ускорить обработку данных и снизить нагрузку на систему. Также был представлен пользовательский интерфейс.
