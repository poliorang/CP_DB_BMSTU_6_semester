\section*{ЗАКЛЮЧЕНИЕ}
\addcontentsline{toc}{section}{ЗАКЛЮЧЕНИЕ}

В ходе выполнения данной работы были выполнены следующие задачи:

\begin{itemize}[label=---]
	\item формализована задача и определен необходимый функционал;
	\item описана структура объектов базы данных, а также сделан выбор СУБД для ее хранения и взаимодействия;
	\item спроектирован и разработан триггер, осуществляющий перерасчёт очков этапов, участников и команд при добавлении улова;
	\item спроектирован и разработан интерфейс взаимодействия с базой данных;
	\item проведено исследование времени работы операции добавления улова, а также функции получения списка спортсменов соревнования в зависимости от количества участников.
\end{itemize}

Была достигнута поставленная цель: спроектирована и разработана база данных для проведения соревнований Российской Федерации Подводного Рыболовства, интерфейсом для которой стало приложения для iOS. Необходимый функционал был реализован.
Также в ходе исследования было выявлено, что при изменении количества участников в 100 раз (от 10 до 1000), время, затрачиваемое на работу функции получения списка участников, увеличивается в 139 раз, а время, затрачиваемое на работу функции создания улова, увеличивается лишь в 42 раза.

\pagebreak